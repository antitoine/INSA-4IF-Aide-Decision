%%% Preamble
\documentclass[paper=a4, fontsize=11pt]{scrartcl}
\usepackage[utf8]{inputenc}
\usepackage[T1]{fontenc}
\usepackage{fourier}
\usepackage[french]{babel}
\usepackage[protrusion=true,expansion=true]{microtype}	
\usepackage{amsmath,amsfonts,amsthm} % Math packages
\usepackage[pdftex]{graphicx}	
\usepackage{url}
\usepackage{pdfpages}
\usepackage{todonotes}
\usepackage[a4paper, body={17cm,25cm}]{geometry}
\usepackage{float}
\usepackage{framed}


%%% Custom sectioning
\usepackage{sectsty}
\allsectionsfont{  \normalfont\scshape}
%\allsectionsfont{\centering \normalfont\scshape}

%%% Custom headers/footers (fancyhdr package)
\usepackage{fancyhdr}
\pagestyle{fancyplain}
\fancyhead{}											% No page header
\fancyfoot[L]{}											% Empty 
\fancyfoot[C]{}											% Empty
\fancyfoot[R]{\thepage}									% Pagenumbering
\renewcommand{\headrulewidth}{0pt}			% Remove header underlines
\renewcommand{\footrulewidth}{0pt}				% Remove footer underlines
\setlength{\headheight}{13.6pt}


%%% Equation and float numbering
\numberwithin{equation}{section}		% Equationnumbering: section.eq#
\numberwithin{figure}{section}			% Figurenumbering: section.fig#
\numberwithin{table}{section}				% Tablenumbering: section.tab#


%%% Define new commands
\newcommand{\horrule}[1]{\rule{\linewidth}{#1}} 	% Horizontal rule
\renewcommand{\bf}[1]{\textbf{#1}}
\renewcommand{\it}[1]{\textit{#1}}

\newcommand{\Todo}[1]{\todo[inline]{#1}}

\usepackage{color}
\usepackage{xcolor}

\usepackage{caption}
\DeclareCaptionFont{white}{\color{white}}
\DeclareCaptionFormat{listing}{\colorbox{gray}{\parbox{\textwidth}{#1#2#3}}}
\captionsetup[lstlisting]{format=listing,labelfont=white,textfont=white}

\usepackage[numbered]{mcode}

\lstset{breaklines=true,columns=fullflexible}
\colorlet{shadecolor}{black!10}


%%% Begin document
\begin{document}
\includepdf[pages={1}]{title.pdf}

\tableofcontents
\listoftodos

\newpage

%%%%%%%%%%%%%%%%%%%%%%%%%%%%%%%%%%%%%%%%%%%%%%%%%%%%%%%%%%%%%%%%%%%%%%%%%%%%%%%%%%%%%%%%%%%%%%%%%%%%%%%%%%%%%%%%%%%%%%%%%%%
%%%%%%%%%%%%%%%%%%%%%%%%%%%%%%%%%%%%%%%%%%%%%%%%%%%%%%%%%%%%%%%%%%%%%%%%%%%%%%%%%%%%%%%%%%%%%%%%%%%%%%%%%%%%%%%%%%%%%%%%%%%
%%%%%%%%%%%%%%%%%%%%%%%%%%%%%%%%%%%%%%%%%%%%%%%%%%%%%%%%%%%%%%%%%%%%%%%%%%%%%%%%%%%%%%%%%%%%%%%%%%%%%%%%%%%%%%%%%%%%%%%%%%%
\section{\'Etude des problèmes soumis par les responsables}
Cette section traite de la résolution des problèmes soumis par les différents acteurs de l'entreprise. Chaque sous-section aborde un problème et tente de proposer une solution optimale plausible pour ce dernier.

%%%%%%%%%%%%%%%%%%%%%%%%%%%%%%%%%%%%%%%%%%%%%%%%%%%%%%%% NOTATIONS %%%%%%%%%%%%%%%%%%%%%%%%%%%%%%%%%%%%%%%%%%%%%%%%%%%%%%%%

\begin{shaded}
\vspace{-0.5cm}

\subsection*{Notations}
Dans le document suivant, nous allons utiliser : 
$A$, $B$, $C$, $D$, $E$ et $F$ pour désigner les différents produits $P_i$. Les quantités respectives $q_i$ seront désignées par $q_A$, $q_B$, $q_C$, $q_D$, $q_E$ et $q_F$. Le prix de vente du produit $P_i$ sera noté $p_i$. Le coût de production du produit $P_i$ sera noté $c_i$, le coût de production comprends le coût des matières premières et le coût de fabrication dû aux machines.\\

Le vecteur des quantités $(q_A$, $q_B$, $q_C$, $q_D$, $q_E$, $q_F)$ sera noté $Q$ dans la suite du rapport.

\end{shaded}
%%%%%%%%%%%%%%%%%%%%%%%%%%%%%%%%%%%%%%%%%%%%%%%%%%%%%%%% COMPTABLE %%%%%%%%%%%%%%%%%%%%%%%%%%%%%%%%%%%%%%%%%%%%%%%%%%%%%%%%
\subsection{Comptable}
\bf{Problème soumis :}
Le problème soumis par le comptable est le suivant, celui-ci souhaite maximiser les bénéfices de l'entreprise.\\

\bf{Solution proposée :}

On a cherché à maximiser le bénéfice que l'on a exprimé, 

\[f(Q) = (\sum_i q_i \times p_i ) - (\sum_i q_i \times c_i) \quad \text{  avec } \, Q, \, \text{ vecteur des quantités } \, q_i \]

ce qui nous donne pour chaque produit la quantité hebdomadaire à produire :
\begin{center}
\begin{tabular}{cccccc}
\hline
$q_A$ & $q_B$ & $q_C$ & $q_D$ & $q_E$ & $q_F$ \\
\hline
5 & 18 & 0 & 0 & 240 & 93,67 \\
\hline
\end{tabular}
\end{center}

Ce résultat se comprend très rapidement si on regarde la rentabilité de chaque produit présenté ci-dessous :

\begin{center}
\begin{tabular}{cccccc}
\hline
$q_A$ & $q_B$ & $q_C$ & $q_D$ & $q_E$ & $q_F$ \\
\hline
5,83 & 11,61 & 12,16 & 1,3 & 31,65 & 27,48 \\
\hline
\end{tabular}
\end{center}

On remarque que le produit E est le plus rentable, puis le produit F, puis C, puis B, puis A et enfin D.
De ce fait, il faut favoriser la production des produits en suivant cet ordre et donc facorisé largement le produit E à contrario du produit D.

%%%%%%%%%%%%%%%%%%%%%%%%%%%%%%%%%%%%%%%%%%%%%%%%%%%%%%%% RESP ATELIER %%%%%%%%%%%%%%%%%%%%%%%%%%%%%%%%%%%%%%%%%%%%%%%%%%%%%%%%

\subsection{Responsable d'atelier}
\bf{Problème soumis :}

Le problème soumis par le responsable de l'atelier est le suivant, ce dernier souhaite maximiser la quantité totale de produits fabriqués. De ce fait, la fonction à maximiser ne devra pas favoriser un type de produit par rapport à un autre.\\

\bf{Solution proposée :}

On cherche de ce fait à maximiser l'équation (en tenant des comptes des contraintes):

\[f(Q) = \sum_i q_i \quad \text{  avec } \, Q, \, \text{ vecteur des quantités } \, q_i \]

\bf{Résultat :}

Le résultat de l'étude est le suivant : 

\begin{center}
\begin{tabular}{cccccc}
\hline 
$q_A$ & $q_B$ & $q_C$ & $q_D$ & $q_E$ & $q_F$ \\ 
\hline 
5,00 & 54.64 & 38,85 & 0,00 & 181,71 & 98,43 \\ 
\hline 
\end{tabular} 
\end{center}

On constate donc qu'il ne faut pas fabriquer de produits D si l'on souhaite maximiser le nombre de produits à fabriquer. Ceci n'est cependant pas une solution qui sera compatible avec les autres responsables. En effet, il y a un déséquilibre important entre la production de produit $A$ et $E$, ce qui ne conviendra pas au responsable commercial.

Cette étude permet de trouver la quantité maximale de produit qui peut être produite par l'entreprise.
\[ Q_{max} = 378,6 \text{ produits}\]

Cette limite supérieure présente l'intérêt de proposer intervale de recherche pour d'autres études.


%%%%%%%%%%%%%%%%%%%%%%%%%%%%%%%%%%%%%%%%%%%%%%%%%%%%%%%% RESP DES STOCK %%%%%%%%%%%%%%%%%%%%%%%%%%%%%%%%%%%%%%%%%%%%%%%%%%%%%%%%
\subsection{Responsable des stocks}
\bf{Problème soumis :}

Le responsable des stocks souhaite quant à lui minimiser la quantité de matière stockée (qu'elle soit brute ou transformée).\\

Ainsi on doit donc minimiser la quantité de produits stockés (c'est-à-dire à minimiser $q_A + q_B + q_C + q_D + q_E + q_F$, ainsi que les matières premières stockées (soit $4q_A + 4q_B + 5q_C + 9q_D + 4q_E + 3q_F$).
\\

\bf{Solution proposée :}

On souhaite donc minimiser la fonction suivante : 

\[ f(Q) = 5q_A + 5q_B + 6q_C + 10q_D + 5q_E + 4q_F \]

Cependant, sans contrainte supplémentaire, on se place dans le cas irréaliste où l'on pourrait diminuer d'autant que l'on souhaite le nombre de produits fabriqués par l'entreprise.\\
Il faut donc ajouter une contrainte, qui peut être une contrainte sur un nombre minimal de produits à fabriquer, ou bien sur le bénéfice souhaité par exemple.\\
Pour l'étude, il a été fixé un nombre minimal de produit à fabriquer. Nous avons fait varier ce nombre pour voir s'il n'y a pas un "sweet spot".

\bf{Résultat :}

Le résultat de l'étude est le suivant : 

\begin{figure}[H]
\caption{Nombre de produits en fonction du stock total \label{figstock}}
\centering
\includegraphics[width=16cm]{figures/nbProduitsFctStockTotal.png}
\end{figure}

\begin{figure}[H]
\caption{Variation du coefficient de la courbe de la figure \ref{figstock}}
\centering
\includegraphics[width=16cm]{figures/nbProduitsFctStockTotal_Coeff.png}
\end{figure}

\begin{figure}[H]
\caption{Bénéfice réalisé en fonction du stock total \label{figben}}
\centering
\includegraphics[width=16cm]{figures/BenefFctStockTotal.png}
\end{figure}

\begin{figure}[H]
\caption{Variation du coefficient de la courbe de la figure \ref{figben}}
\centering
\includegraphics[width=16cm]{figures/BenefFctStockTotal_Coeff.png}
\end{figure}


Les figures montre que pour une production réaliste de produits fabriqués ($> 200$), on peut voir que la rentabilité par produit chute à partir de 1003 unités de stocks, soit une production de 243 produits dont Matlab nous donne la répartition.

\begin{center}
\begin{tabular}{cccccc}
\hline 
$q_A$ & $q_B$ & $q_C$ & $q_D$ & $q_E$ & $q_F$ \\ 
\hline 
5,00 & 5,00 & 0,00 & 0,00 & 56,5 & 167,5 \\ 
\hline 
\end{tabular} 
\end{center}



%%%%%%%%%%%%%%%%%%%%%%%%%%%%%%%%%%%%%%%%%%%%%%%%%%%%%%%% RESP COMMERCIAL %%%%%%%%%%%%%%%%%%%%%%%%%%%%%%%%%%%%%%%%%%%%%%%%%%%%%%%%
\subsection{Responsable commercial}
\bf{Problème soumis :}

Le responsable commercial souhaite pour sa part équilibrer la quantité de produite des produits A et E.\\

\bf{Solution proposée :}

Suite à notre étude du problème, c'est à dire la maximisation des bénéfices en équilibrant la production des produits A et E, plusieurs constats peuvent être faits :\\
Pour un écart nul entre les quantités de A et E produites on a les quantités hebdomadaires suivantes suivants :
\begin{center}
\begin{tabular}{cccccc}
\hline
$q_A$ & $q_B$ & $q_C$ & $q_D$ & $q_E$ & $q_F$ \\
\hline
119.08 & 6.91 & 42.58 & 0.00 & 119.08 & 87.24 \\
\hline
\end{tabular}
\end{center}
Tout d'abord le produit D ne semble pas rentable car l'étude conclue qu'il serait optimal de ne plus le produire du tout.
On remarque également que l’évolution de e tend à favoriser la production du produit E plutôt que le produit A.

%%%%%%%%%%%%%%%%%%%%%%%%%%%%%%%%%%%%%%%%%%%%%%%%%%%%%%%% RESP DU PERSONNEL %%%%%%%%%%%%%%%%%%%%%%%%%%%%%%%%%%%%%%%%%%%%%%%%%%%%%%%%
\subsection{Responsable du personnel}
\bf{Problème soumis :}

Le responsable du personnel souhaite utiliser le moins possible la machine 4.\\

\bf{Solution proposée :}

La première solution qui nous vient à l’esprit serait de minimiser la fonction suivante :\[f(Q) = 2q_A + 10q_B + 5q_C + 4q_D + 13q_E + 7q_F\]

Cependant, les résultats que l’on obtient en ne prenant en compte que cette équation ne sont pas satisfaisant, car on se retrouverait à ne produire que 5 unités de A et B et 0 des autres. Ce résultat est mathématiquement juste au vu des contraintes que nous avions précisé, mais ne suffit pas pour garder une rentabilité à l’entreprise. Pour trouver une solution plus juste, nous allons prendre en compte une certaine quantité d’unités minimales à produire. \\

Nous reprenons donc les résultats obtenus par le responsable d’atelier, et en faisant varier les différentes quantités de produits fabriqués selon sa méthode de calcul, nous avons établi un graphique permettant d’évaluer le temps d’utilisation de la machine 4.\\

\begin{figure}[H]
\caption{Temps d'utilisation de la machine 4 en fonction du nombre de produits fabriqués}
\centering
\includegraphics[width=16cm]{figures/graphe-personnel.png}
\end{figure}

On constate très clairement que la pente de la courbe augmente fortement à partir de 345 unités fabriquée, ce qui correspond à 940 min de travail pour la machine 4. Au delà de ce seuil, on utilise beaucoup plus la machine 4.\\

345 étant la limite à jusqu'à laquelle on peut produire le plus d’unités en utilisant le moins la machine 4, on choisira donc arbitrairement un minimum de 345 unités à fabriquer.

\begin{center}
\begin{tabular}{cccccc}
\hline
$q_A$ & $q_B$ & $q_C$ & $q_D$ & $q_E$ & $q_F$ \\
\hline
270.00 & 5.00 & 70.00 & 0.00 & 0.00 & 0.00 \\
\hline
\end{tabular}
\end{center}


\section{Bilan}

Voici le tableau récapitulatif pour les différents acteurs de l'entreprise :

\begin{center}
\begin{tabular}{l|cccccc|c}
\hline
Responsables & $q_A$ & $q_B$ & $q_C$ & $q_D$ & $q_E$ & $q_F$ & $\sum q_i \quad$ \\
\hline
Comptable & 5 & 18 & 0 & 0 & 240 & 93,67 & 356,67 \\
Responsable d'atelier & 5,00 & 54.64 & 38,85 & 0,00 & 181,71 & 98,43 & 378,63 \\
Responsable des stocks & 5,00 & 5,00 & 0,00 & 0,00 & 56,5 & 167,5 & 234,00 \\
Responsable commercial & 119.08 & 6.91 & 42.58 & 0.00 & 119.08 & 87.24 & 374,89 \\
Responsable personnel & 270.00 & 5.00 & 70.00 & 0.00 & 0.00 & 0.00 & 345,00 \\

\hline
\end{tabular}
\end{center}
%%%%%%%%%%%%%%%%%%%%%%%%%%%%%%%%%%%%%%%%%%%%%%%%%%%%%%%%%%%%%%%%%%%%%%%%%%%%%%%%%%%%%%%%%%%%%%%%%%%%%%%%%%%%%%%%%%%%%%%%%%%%%%%%
%%%%%%%%%%%%%%%%%%%%%%%%%%%%%%%%%%%%%%%%%%%%%%%%%%%%%%%% RESP DES STOCK %%%%%%%%%%%%%%%%%%%%%%%%%%%%%%%%%%%%%%%%%%%%%%%%%%%%%%%%
%%%%%%%%%%%%%%%%%%%%%%%%%%%%%%%%%%%%%%%%%%%%%%%%%%%%%%%%%%%%%%%%%%%%%%%%%%%%%%%%%%%%%%%%%%%%%%%%%%%%%%%%%%%%%%%%%%%%%%%%%%%%%%%%
\section{\'Etude du problème soumis par le Responsable de l'entreprise}

\newpage

\section{Listing - Scripts Matlab}

\lstlistoflistings

\lstinputlisting[caption=Comptable]{scripts/Comptable.m}
\lstinputlisting[caption=Responsable Atelier]{scripts/ResponsableAtelier.m}
\lstinputlisting[caption=Responsable Commercial]{scripts/ResponsableCommercial.m}
\lstinputlisting[caption=Responsable Stocks]{scripts/ResponsableStock.m}
\lstinputlisting[caption=Responsable Personel]{scripts/ResponsablePersonnel.m}

%%% End document
\end{document}
