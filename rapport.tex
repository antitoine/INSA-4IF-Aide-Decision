%%% Preamble
\documentclass[paper=a4, fontsize=11pt]{scrartcl}
\usepackage[utf8]{inputenc}
\usepackage[T1]{fontenc}
\usepackage{fourier}
\usepackage[french]{babel}
\usepackage[protrusion=true,expansion=true]{microtype}	
\usepackage{amsmath,amsfonts,amsthm} % Math packages
\usepackage[pdftex]{graphicx}	
\usepackage{url}
\usepackage{pdfpages}


%%% Custom sectioning
\usepackage{sectsty}
\allsectionsfont{  \normalfont\scshape}
%\allsectionsfont{\centering \normalfont\scshape}

%%% Custom headers/footers (fancyhdr package)
\usepackage{fancyhdr}
\pagestyle{fancyplain}
\fancyhead{}											% No page header
\fancyfoot[L]{}											% Empty 
\fancyfoot[C]{}											% Empty
\fancyfoot[R]{\thepage}									% Pagenumbering
\renewcommand{\headrulewidth}{0pt}			% Remove header underlines
\renewcommand{\footrulewidth}{0pt}				% Remove footer underlines
\setlength{\headheight}{13.6pt}


%%% Equation and float numbering
\numberwithin{equation}{section}		% Equationnumbering: section.eq#
\numberwithin{figure}{section}			% Figurenumbering: section.fig#
\numberwithin{table}{section}				% Tablenumbering: section.tab#


%%% Define new commands
\newcommand{\horrule}[1]{\rule{\linewidth}{#1}} 	% Horizontal rule
\renewcommand{\bf}[1]{\textbf{#1}}
\renewcommand{\it}[1]{\textit{#1}}

%%% Begin document
\begin{document}
\includepdf[pages={1}]{title.pdf}

\tableofcontents
\newpage

%%%%%%%%%%%%%%%%%%%%%%%%%%%%%%%%%%%%%%%%%%%%%%%%%%%%%%%%%%%%%%%%%%%%%%%%%%%%%%%%%%%%%%%%%%%%%%%%%%%%%%%%%%%%%%%%%%%%%%%%%%%
%%%%%%%%%%%%%%%%%%%%%%%%%%%%%%%%%%%%%%%%%%%%%%%%%%%%%%%%%%%%%%%%%%%%%%%%%%%%%%%%%%%%%%%%%%%%%%%%%%%%%%%%%%%%%%%%%%%%%%%%%%%
%%%%%%%%%%%%%%%%%%%%%%%%%%%%%%%%%%%%%%%%%%%%%%%%%%%%%%%%%%%%%%%%%%%%%%%%%%%%%%%%%%%%%%%%%%%%%%%%%%%%%%%%%%%%%%%%%%%%%%%%%%%
\section{\'Etude des problèmes soumis par les responsables}
Cette section traite de la résolution des problèmes soumis par les différents acteurs de l'entreprise. Chaque sous-section aborde un problème et tente de proposer une solution optimale plausible pour ce dernier.

%%%%%%%%%%%%%%%%%%%%%%%%%%%%%%%%%%%%%%%%%%%%%%%%%%%%%%%% NOTATIONS %%%%%%%%%%%%%%%%%%%%%%%%%%%%%%%%%%%%%%%%%%%%%%%%%%%%%%%%
\subsection*{Notations}

Dans le document suivant, nous allons utiliser : 
$A$, $B$, $C$, $D$, $E$ et $F$ pour désigner les différents produits $P_i$. Les quantités respectives $q_i$ seront désignées par $q_A$, $q_B$, $q_C$, $q_D$, $q_E$ et $q_F$. Le prix de vente du produit $P_i$ sera noté $p_i$. Le coût de production du produit $P_i$ sera noté $c_i$, le coût de production comprends le coût des matières premières et le coût de fabrication dû aux machines.

%%%%%%%%%%%%%%%%%%%%%%%%%%%%%%%%%%%%%%%%%%%%%%%%%%%%%%%% COMPTABLE %%%%%%%%%%%%%%%%%%%%%%%%%%%%%%%%%%%%%%%%%%%%%%%%%%%%%%%%
\subsection{Comptable}
\bf{Problème soumis :}
Le problème soumis par le comptable est le suivant, celui-ci souhaite maximiser les bénéfices de l'entreprise.\\

\bf{Solution proposée :}

On a cherché à maximiser le bénéfice que l'on a exprimé, 

\[f(Q) = (\sum_i q_i \times p_i ) - (\sum_i q_i \times c_i) \quad \text{  avec } \, Q, \, \text{ vecteur des quantités } \, q_i \]

ce qui nous donne pour chaque produit la quantité hebdomadaire à produire :
\begin{center}
\begin{tabular}{cccccc}
\hline
$q_A$ & $q_B$ & $q_C$ & $q_D$ & $q_E$ & $q_F$ \\
\hline
5,83 & 11,61 & 12,16 & 1,3 & 31,65 & 27,48 \\
\hline
\end{tabular}
\end{center}

On remarque que le produit E est le plus rentable, puis le produit F, puis C, puis B, puis A et enfin D. De ce fait, il faut favoriser la production des produits en suivant cet ordre. Ce qui explique que le produit E est celui qui doit être le plus produit et le produit D est celui qui doit être le moins produit.

%%%%%%%%%%%%%%%%%%%%%%%%%%%%%%%%%%%%%%%%%%%%%%%%%%%%%%%% RESP ATELIER %%%%%%%%%%%%%%%%%%%%%%%%%%%%%%%%%%%%%%%%%%%%%%%%%%%%%%%%

\subsection{Responsable d'atelier}
\bf{Problème soumis :}

Le problème soumis par le responsable de l'atelier est le suivant, ce dernier souhaite maximiser la quantité totale de produits fabriqués. De ce fait, la fonction à maximiser ne devra pas favoriser un type de produit par rapport à un autre.\\

\bf{Solution proposée :}

On cherche de ce fait à maximiser l'équation (en tenant des comptes des contraintes):

\[f(Q) = \sum_i q_i \quad \text{  avec } \, Q, \, \text{ vecteur des quantités } \, q_i \]

\bf{Résultat :}

Le résultat de l'étude est le suivant : 

\begin{center}
\begin{tabular}{cccccc}
\hline 
$q_A$ & $q_B$ & $q_C$ & $q_D$ & $q_E$ & $q_F$ \\ 
\hline 
5 & 34,64 & 38,85 & 0 & 181,71 & 98,43 \\ 
\hline 
\end{tabular} 
\end{center}

On constate donc qu'il ne faut pas fabriquer de produits D si l'on souhaite maximiser le nombre de produits à fabriquer. Ceci n'est cependant pas une solution qui sera compatible avec les autres responsables. En effet, il y a un déséquilibre important entre la production de produit $A$ et $E$, ce qui ne conviendra pas au responsable commercial.

Cette étude permet de trouver la quantité maximale de produit qui peut être produite par l'entreprise.
\[ Q_{max} = 378,6 \text{ produits}\]

Cette limite supérieure présente l'intérêt de proposer intervale de recherche pour d'autres études.


%%%%%%%%%%%%%%%%%%%%%%%%%%%%%%%%%%%%%%%%%%%%%%%%%%%%%%%% RESP DES STOCK %%%%%%%%%%%%%%%%%%%%%%%%%%%%%%%%%%%%%%%%%%%%%%%%%%%%%%%%
\subsection{Responsable des stocks}
\bf{Problème soumis :}

Le responsable des stocks souhaite quant à lui minimiser la quantité de matière stockée (qu'elle soit brute ou transformée).\\

\bf{Solution proposée :}

%%%%%%%%%%%%%%%%%%%%%%%%%%%%%%%%%%%%%%%%%%%%%%%%%%%%%%%% RESP COMMERCIAL %%%%%%%%%%%%%%%%%%%%%%%%%%%%%%%%%%%%%%%%%%%%%%%%%%%%%%%%
\subsection{Responsable commercial}
\bf{Problème soumis :}

Le responsable commercial souhaite pour sa part équilibrer la quantité de produite des produits A et E.\\

\bf{Solution proposée :}

Suite à notre étude du problème, c'est à dire la maximisation des bénéfices en équilibrant la production des produits A et E, plusieurs constats peuvent être faits :\\
Pour un écart nul entre les quantités de A et E produites on a les quantités hebdomadaires suivantes suivants :
\begin{center}
\begin{tabular}{cccccc}
\hline
$q_A$ & $q_B$ & $q_C$ & $q_D$ & $q_E$ & $q_F$ \\
\hline
119.08 & 6.91 & 42.58 & 0.00 & 119.08 & 87.24 \\
\hline
\end{tabular}
\end{center}
Tout d'abord le produit D ne semble pas rentable car l'étude conclue qu'il serait optimal de ne plus le produire du tout.
On remarque également que l’évolution de e tend à favoriser la production du produit E plutôt que le produit A.

%%%%%%%%%%%%%%%%%%%%%%%%%%%%%%%%%%%%%%%%%%%%%%%%%%%%%%%% RESP DU PERSONNEL %%%%%%%%%%%%%%%%%%%%%%%%%%%%%%%%%%%%%%%%%%%%%%%%%%%%%%%%
\subsection{Responsable du personnel}
\bf{Problème soumis :}

Le responsable du personnel souhaite minimiser le temps d'utilisation de la machine 4.\\

\bf{Solution proposée :}

%%%%%%%%%%%%%%%%%%%%%%%%%%%%%%%%%%%%%%%%%%%%%%%%%%%%%%%%%%%%%%%%%%%%%%%%%%%%%%%%%%%%%%%%%%%%%%%%%%%%%%%%%%%%%%%%%%%%%%%%%%%%%%%%
%%%%%%%%%%%%%%%%%%%%%%%%%%%%%%%%%%%%%%%%%%%%%%%%%%%%%%%% RESP DES STOCK %%%%%%%%%%%%%%%%%%%%%%%%%%%%%%%%%%%%%%%%%%%%%%%%%%%%%%%%
%%%%%%%%%%%%%%%%%%%%%%%%%%%%%%%%%%%%%%%%%%%%%%%%%%%%%%%%%%%%%%%%%%%%%%%%%%%%%%%%%%%%%%%%%%%%%%%%%%%%%%%%%%%%%%%%%%%%%%%%%%%%%%%%
\section{\'Etude du problème soumis par le Responsable de l'entreprise}

%%% End document
\end{document}
